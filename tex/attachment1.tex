\footnotesize{
\chapter{EPOCH input file}
\label{att:input-deck}
\begin{verbatim}
	begin:constant
	konstanta1 = 1e19
	konstanta2 = 2.0
	konstanta3 = 60
	
	las_lambda    = 0.8 * micron
	ll            = las_lambda
	las_intensity_focus_w_cm2 = konstanta1
	a0            = ll / micron * sqrt(las_intensity_focus_w_cm2/1.37e18)
	gamma0        = sqrt(1.0+a0^2)
	pulse_length  = 30.0 * femto
	spot_size     = 3.2 * micron
	cpwl          = 30
	simtime       = 800.0 * femto
	theta         = konstanta3 / 180.0 * pi
	ct            = cos(theta)
	st            = sin(theta)
	n_el_over_nc  = 3.0*gamma0
	min_density   = 0.01
	preplasma_length = konstanta2 * micron
	foil_thickness = 4.0 * micron
	profile_thickness = - loge(min_density) * preplasma_length
	boundary_thickness = 1.0 * micron
	x1 = 60 * micron - foil_thickness - 
	    profile_thickness - boundary_thickness
	x2 = x1 + profile_thickness
	x3 = x2 + foil_thickness
	xfocus        = x2
	yc            = - xfocus * tan(theta) * 0.5
	sigma_to_fwhm = 2.0 * sqrt(loge(2.0))
	w0            = spot_size / sigma_to_fwhm
	zR            = pi * w0^2 / ll
	lfocus        = xfocus / ct
	lfzr          = lfocus / zR
	wl            = w0 * sqrt(1.0+lfzr^2)
	intensity_fac2d = 1.0 / sqrt(1.0+lfzr^2) * ct
	las_omega     = 2.0 * pi* 299792458.0 / ll
	las_time      = 2.0 * pi / las_omega
	wt            = pulse_length / sigma_to_fwhm
	n_crit        = critical(las_omega)
	n_el          = n_el_over_nc * n_crit
	y_full        = 54.0 * micron
	end:constant
	
	begin:control
	x_min = 0.0 * micron
	x_max = 60.0 * micron
	y_min = -55.0 * micron
	y_max = 55.0 * micron 
	nx = (x_max-x_min) / ll * cpwl
	ny = (y_max-y_min) / ll * cpwl
	t_end = simtime
	particle_tstart = x1 / ct / 299792458.0
	dt_multiplier = 0.97
	dlb_threshold = 0.5
	use_random_seed = T
	stdout_frequency = 10
	smooth_currents = T
	field_order = 2
	end:control
	
	begin:boundaries
	cpml_thickness = 6
	cpml_kappa_max = 20
	cpml_a_max = 0.15
	cpml_sigma_max = 0.7
	bc_x_min = cpml_laser
	bc_x_max_field = cpml_outflow
	bc_y_min_field = periodic
	bc_y_max_field = periodic
	bc_x_min_particle = heat_bath
	bc_x_max_particle = heat_bath
	bc_y_min_particle = heat_bath
	bc_y_max_particle = heat_bath
	end:boundaries
	
	begin:laser
	boundary = x_min
	intensity_w_cm2 = las_intensity_focus_w_cm2 * intensity_fac2d
	lambda = ll
	t_profile = gauss(time, pulse_length, wt)
	t_end = 3.0 * pulse_length
	phase = -2.0*pi*(y-yc)*st/ll+((pi/ll)*(((y-yc)*ct)^2))/
	    (lfocus*(1.0+(1.0/lfzr)^2))-atan((y-yc)*st/zR+lfzr)
	profile = gauss((y-yc)*ct,0.0,wl)
	end:laser
	
	begin:species
	name = electron
	charge = -1.0
	mass = 1.0
	dump = T
	npart_per_cell = 30 
	density = if((x gt x1) and (x lt x2) and (abs(y) lt y_full),
	    n_el*exp((x-x2)/preplasma_length), 0.0)
	density = if((x gt x2) and (x lt x3) and (abs(y) lt y_full), n_el,
	    density(electron))
	temp_ev = 100.0
	identify:electron
	end:species
	
	begin:species
	name = proton
	charge = 1.0
	mass = 1836.0
	dump = T
	npart_per_cell = 30 
	density = density(electron)
	temp_ev = 100.0
	end:species
	
	begin:probe
	name = electron_back_probe
	point = (x3-2.0*micron, 0.0)
	normal = (1.0, 0.0)
	ek_min = 1.602e-16
	ek_max = -1.0
	include_species:electron
	dumpmask = always
	end:probe
	
	#begin:output
	#  dt_snapshot = 10.0*femto
	#  name = fields
	#  file_prefix = f
	#  restartable = F
	#  grid = always + single
	#  ex = always + single
	#  ey = always + single
	#  bz = always + single
	#end:output
	
	#begin:output
	#  dt_snapshot = 10.0*femto
	#  name = plasma
	#  file_prefix = n
	#  restartable = F
	#  grid = always
	#  number_density = always + single
	#  poynt_flux = always + single
	#  absorption = always
	#  total_energy_sum = always
	#end:output
	
	begin:output
	dt_snapshot = 700.0*femto
	name = prob
	file_prefix = ppr
	restartable = F
	particle_probes = always
	end:output
\end{verbatim}
}