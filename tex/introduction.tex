\chapter*{Introduction}
\addcontentsline{toc}{chapter}{Introduction}


Plasma, the fourth state of matter, makes up more than 99\% of the visible universe~\cite{plasma-intro}. It has been observed in technology since 1810 in various cases of electrical discharge, but a deeper study of plasma has begun in the 1950s in the context of a controlled nuclear fusion. Later, with the development of high-power lasers, it became possible to create a plasma with densities similar to those of solids.

Advancements in laser technology have enabled us to study plasma under increasingly extreme conditions. In parallel, computer simulations offer a complementary method, allowing us to explore specific effects from the first principles. However, for some applications, surrogate models may be more appropriate. These models focus solely on describing the relationship between inputs and outputs, without incorporating the underlying physical principles.

In this thesis, we create a surrogate model for laser absorption in high-intensity short-pulse laser-plasma interactions. We explore a segment of the parameter space using several hundred particle-in-cell simulations and model the hot electron temperature based on these parameters. This multi-step process is presented in detail throughout the thesis.

In the first chapter, we introduce the fundamental physical effects relevant to the studied problem, providing the necessary background to understand the data obtained from simulations. This chapter also covers the motivation behind this work and discusses key aspects of particle-in-cell simulations.

Chapter 2 focuses on retrieving the hot electron temperature from the simulation results. We aim for an unsupervised process, which facilitates the addition of more simulations in the future by automating the procedure. Various fitting methods are reviewed for estimating the parameters without supervision and one explicit method is explained.

In Chapter 3, we discuss three potential surrogate models suitable for this thesis. Although we do not implement these models ourselves, we explain the underlying principles on which they are based.

The fourth chapter is dedicated to implementing the fitting method introduced in Chapter 2. We address the challenges encountered when processing large amounts of simulation data, examine the strengths and weaknesses of the proposed fitting method, and compare the results to what we consider reliable estimates of fit parameters.

Chapter 5, the final chapter, centers on applying the models discussed in Chapter 3. We compare the performance of these models on the dataset analyzed in Chapter 4, we present a tool for visually inspecting the models and we propose a strategy for expanding the dataset by running additional simulations.

This comprehensive approach aims to create a robust surrogate model for better understanding laser-plasma interactions in certain part of the parameter space with possible application in optimizing the X-ray photon yields.  We try to maximize the use of both simulation data and advanced modelling techniques.

