\documentclass[a4paper,oneside,12pt]{book}
%% === nezbytné balíčky:
\usepackage[IL2]{fontenc}   % písmo pro UTF-8 || jiné: [T1] (funguje pro UTF-8?)
\usepackage[utf8]{inputenc} % vstupní znaková sada UTF-8 || [cp1250] -> Windows 1250 || [latin2] -> ISO Latin 2

\usepackage[czech]{babel} % česky psaná práce, typografická pravidla     

\usepackage{pdfpages} % pokud nemáte formulář "Zadání bak./dipl. práce" naskenovaný jako PDF, tak ZAKOMENTUJTE

%\usepackage{encxvlna} % postará se o spojky a předložky, které dle českých pravidel nesmí být na konci řádku. Dokumentace: http://texdoc.net/texmf-dist/doc/generic/encxvlna/encxvlna.pdf

\usepackage[a4paper, hmarginratio=1:1]{geometry} % využití celé A4 stránky a nastavení okrajů, pro OBOUSTRANNÝ TISK

%% === balíčky, které se mohou hodit:
\usepackage[hidelinks]{hyperref} % v PDF budou klikací odkazy ("hidelinks" je nebude rámovat)
\usepackage{graphicx} % balíček pro vkládání RASTROVÝCH grafických souborů (PNG apod.)
%\usepackage{epsfig} % balíčky pro vkládání VEKTOROVÝCH grafických souborů typu EPS

%\usepackage{float} % rozšířené možnosti umístění obrázků
%\usepackage{caption} % pro popisky obrázků, tabulek atd.

\usepackage{tabularx} % rozšířené možnosti tabulek
%\usepackage{tabu} % jiný balík pro rozšířené možnosti tabulek

\usepackage{listings}  % balíček vhodný pro ukázky kódu 
\usepackage{amsmath} % balíček pro pokročilou matematickou sazbu
%\usepackage{color} % pro možnost barevného textu
%\usepackage{fancybox} % umožňuje pokročilé rámečkování
%\usepackage{index} % nutno použít v případě tvorby rejstříku balíčkem makeindex
%\newindex{default}{idx}{ind}{Rejstřík} % zavádí rejstřík v případě použití balíku index


\topmargin=-15mm      % horní okraj trochu menší
\textwidth=150mm      % šířka textu na stránce
\textheight=240mm     % "výška" textu na stránce

\frenchspacing % za větou bude mezislovní mezera (v anglických textech je mezera za větou delší)
\widowpenalty=1000 % "síla" zákazu vdov (= jeden řádek odstavce na konci stránky)
\clubpenalty=1000 % "síla" zákazu sirotků (= jeden řádek/slovo odstavce samostatně na začátku stránky)
\brokenpenalty=1000 % "síla" zákazu zlomu stránky za řádkem, který má na konci rozdělené slovo

\pagenumbering{arabic} % číslování stránek arabskými číslicemi
\pagestyle{plain}      % stránky číslované dole uprostřed

\parindent=0pt % odsazení 1. řádku odstavce
\parskip=7pt   % mezera mezi odstavci

\newcommand{\ti}{\textit} % zkrácený příkaz pro kurzívu
\newcommand{\tb}{\textbf} % zkrácený příkaz pro tučné písmo


%% --- zde jsou makra, tj. "konstanty" - některé musíte změnit! --- %%
\newcommand{\cvut}{České vysoké učení technické v~Praze}
\newcommand{\fjfi}{Fakulta jaderná a fyzikálně inženýrská}
\newcommand{\katedra}{Katedra softwarového inženýrství}
\newcommand{\program}{Aplikace informatiky v~přírodních vědách} % změňte, pokud máte jiný stud. program
\newcommand{\spec}{--} % změňte, pokud studijní program má specializaci

\newcommand{\druh}{Diplomová práce} % nebo "Diplomová práce"
\newcommand{\woman}{} % pokud jste ŽENA, ZMĚŇTE na: ...{\woman}{a} (je to do Prohlášení)

\newcommand{\logoCVUT}{\includegraphics{symbol_cvut_konturova_verze_cb.pdf}} % logo ČVUT -- podle grafického manuálu ČVUT platného od prosince 2016. Pokud nevyhovuje PDF-verze, tak použijte jinou variantu loga: https://www.cvut.cz/logo-a-graficky-manual -> "Symbol a logo ČVUT v Praze"). Pokud chcete logo úplně vynechat, zadejte místo "\includegraphics{...}" text "\vspace{35mm}"

% přesně podle formuláře "Zadání bak./dipl. práce" VYPLŇTE:
\newcommand{\nazevcz}{Český název práce}    % český název práce (přesně podle zadání!)
\newcommand{\nazeven}{Title}          % anglický název práce (přesně podle zadání!)
\newcommand{\autor}{Jméno Příjmení}   % vyplňte své jméno a příjmení (s akademickým titulem, máte-li jej)
\newcommand{\vedouci}{JménoV PříjmeníV} % vyplňte jméno a příjmení vedoucího práce, včetně titulů, např.: Doc. Ing. Ivo Malý, Ph.D.
\newcommand{\pracovisteVed}{\katedra, \fjfi, \cvut} % ZMĚŇTE, pokud vedoucí Vaší práce není z KSI
\newcommand{\konzultant}{--} % POKUD MÁTE určeného konzultanta, NAPIŠTE jeho jméno a příjmení + tituly
\newcommand{\pracovisteKonz}{--} % POKUD MÁTE konzultanta, NAPIŠTE jeho pracoviště

% podle skutečnosti VYPLŇTE:
\newcommand{\rok}{2023}  % rok odevzdání práce (jen rok odevzdání, nikoli celý akademický rok!)
\newcommand{\kde}{Praze} % studenti z Děčína ZMĚNÍ na: "Děčíně" (doplní se k "prohlášení")

\newcommand{\klicova}{Klíčová slova}   % zde NAPIŠTE česky cca 3-5 klíčových slov
\newcommand{\keyword}{Key words}       % zde NAPIŠTE anglicky cca 3-5 klíčových slov (přeložte z češtiny, ale odborně)
\newcommand{\abstrCZ}{Popis práce česky}    % zde NAPIŠTE abstrakt v češtině (alespoň 7 vět, min. 80 slov. Pokuste se, aby CZ i EN abstrakt nezpůsobily přetečení strany 6 na stranu 7, tj. aby se obojí i s klíčovými slovy vešlo na JEDNU stránku)
\newcommand{\abstrEN}{Popis práce anglicky} % zde NAPIŠTE abstrakt v angličtině

\newcommand{\prohlaseni}{Prohlašuji, že jsem svou bakalářskou práci vypracoval\woman{} samostatně a použil\woman{} jsem pouze podklady (literaturu, projekty, SW atd.) uvedené v přiloženém seznamu.} % text prohlášení můžete mírně upravit :-)

\newcommand{\podekovani}{Děkuji ... za ...} % NAPIŠTE poděkování, např. svému vedoucímu:
%"Děkuji Ing. Eleonoře Krtečkové, Ph.D. za vedení mé bakalářské práce a za podnětné návrhy, které ji obohatily."
% NEBO:
% Děkuji vedoucímu práce doc. Ing. Pafnutijovi Snědldítětikaši, Ph.D. za neocenitelné rady a pomoc při tvorbě bakalářské práce.


\begin{document}
%%%%%%%%%%%% TITULNÍ STRANA -- následujících cca 30 řádků se generuje AUTOMATICKY. Neměňte!!! Výjimka: práce psané v angličtině! %%%%%%%%%%%%
\thispagestyle{empty}

\begin{center}
    {\Large \textsc{\cvut}\\[4mm] \textsc{\fjfi}}\par
    \vspace{4mm}
	\tb{\katedra} \par\vspace{3mm}

    \begin{tabular}{l}
		\tb{Studijní program: \program}\\
		\tb{Specializace: \spec}\\
    \end{tabular}

   \vspace{10mm} \logoCVUT \vspace{15mm} 

   {\huge \tb{\nazevcz}\par}
   \vspace{5mm}   
   {\huge \tb{\nazeven}\par}
   
   \vspace{15mm}
   {\Large \MakeUppercase{\druh}}

   \vfill
   {\large
    \begin{tabular}{ll}
    Vypracoval: & \autor\\
    Vedoucí práce: & \vedouci\\
    Rok: & \rok
    \end{tabular}
   }
\end{center}

%%%%%%%%%%%% ZADÁNÍ PRÁCE %%%%%%%%%%%%
% Zadání (podepsané děkanem atd.) dostanou studenti KSI od sekretářky (nebo ji požádají, např. e-mailem).
\newpage  % SEM NESAHEJTE!
\thispagestyle{empty} % SEM NESAHEJTE!

%% naskenované ZADÁNÍ PRÁCE (nechte odkomentovanou jednu možnost!):
%% --- varianta A1: jednostránkové zadání v PDF:
%\includepdf[pages={1}]{zadani_cele.pdf} % A1) naskenováno do PDF
%% --- varianta A2: jednostránkové zadání v JPG:
%\begin{center}
%     \includegraphics[width=1\textwidth]{zadani1.jpg}
%\end{center}
%% --- varianta B1: dvoustránkové zadání v jediném PDF:
\includepdf[pages={1,2}]{zadani_cele.pdf} % PDF má 2 stránky
%% --- varianta B2: dvoustránkové zadání naskenované jako dvě samostatná jednostránková PDF:
%\includepdf[pages={1}]{zadani1.pdf} % 1. strana zadání v PDF
%\includepdf[pages={1}]{zadani2.pdf} % 2. strana zadání v PDF
%% --- varianta B3: dvoustránkové zadání naskenované jako dva samostatné JPG:
%\begin{center}
%     \includegraphics[width=1\textwidth]{zadani1.jpg} % 1. strana zadání
%\end{center}
%\newpage  
%\thispagestyle{empty}
%\begin{center}
%     \includegraphics[width=1\textwidth]{zadani2.jpg} % 2. strana zadání
%\end{center}
%% --- konec varianty B3


%%%%%%%%%%%% Prohlášení -- SEM NESAHEJTE! Generuje se automaticky z výše nastavených maker \kde{} a \prohlaseni{}. %%%%%%%%%%%%
\newpage % SEM NESAHEJTE!
\thispagestyle{empty}  % SEM NESAHEJTE!

~ % SEM NESAHEJTE!
\vfill % prázdné místo. SEM NESAHEJTE!

\tb{Prohlášení} % SEM NESAHEJTE!

\vspace{1em} % vertikální mezera. SEM NESAHEJTE!
\prohlaseni

\vspace{2em}  % SEM NESAHEJTE!
\hspace{-0.5em}\begin{tabularx}{\textwidth}{X c}  % SEM NESAHEJTE!
V \kde\ dne .................... &........................................ \\	% SEM NESAHEJTE!
	& \autor
\end{tabularx}	% SEM NESAHEJTE!


%%%%%%%%%%%% Poděkování -- tuto stránku můžete celou odstranit %%%%%%%%%%%%
\newpage
\thispagestyle{empty}

~
\vfill % prázdné místo

\tb{Poděkování}

\vspace{1em} % vertikální mezera
\podekovani
\begin{flushright}
\autor
\end{flushright}  % <------- tady končí stránka s poděkováním


%%%%%%%%%%%% ABSTRAKT atp. Je generován AUTOMATICKY podle maker nastavených na začátku souboru) %%%%%%%%%%%% 
\newpage   % SEM NESAHEJTE!
\thispagestyle{empty}   % SEM NESAHEJTE!

% příprava:    (na následujících 8 řádků NESAHEJTE!)
\newbox\odstavecbox
\newlength\vyskaodstavce
\newcommand\odstavec[2]{%
    \setbox\odstavecbox=\hbox{%
         \parbox[t]{#1}{#2\vrule width 0pt depth 4pt}}%
    \global\vyskaodstavce=\dp\odstavecbox
    \box\odstavecbox}
\newcommand{\delka}{120mm} % šířka textů ve 2. sloupci tabulky

% použití přípravy:    % dovnitř "tabular" vůbec NESAHEJTE!
\begin{tabular}{ll}
  {\em Název práce:} & ~ \\
  \multicolumn{2}{l}{\odstavec{\textwidth}{\bf \nazevcz}} \\[1em]
  {\em Autor:} & \autor \\[1em]
  {\em Studijní program:} & \program \\
  {\em Specializace:} & \spec \\
  {\em Druh práce:} & \druh \\[1em]
  {\em Vedoucí práce:} & \odstavec{\delka}{\vedouci\\ \pracovisteVed} \\
  {\em Konzultant:} & -- %\odstavec{\delka}{\konzultant \\ \pracovisteKonz}  % VYMAŽTE text "-- %" v případě, že jste neměli konzultanta
 \\[1em]  
  \multicolumn{2}{l}{\odstavec{\textwidth}{{\em Abstrakt:} ~ \abstrCZ  }} \\[1em]
  {\em Klíčová slova:} & \odstavec{\delka}{\klicova} \\[2em]

  {\em Title:} & ~\\
  \multicolumn{2}{l}{\odstavec{\textwidth}{\bf \nazeven}}\\[1em]
  {\em Author:} & \autor \\[1em]
  \multicolumn{2}{l}{\odstavec{\textwidth}{{\em Abstract:} ~ \abstrEN  }} \\[1em]
  {\em Key words:} & \odstavec{\delka}{\keyword}
\end{tabular}



%%%%%%%%%%%% Obsah práce ... je generován AUTOMATICKY %%%%%%%%%%%%
\newpage  % SEM NESAHEJTE!
\parskip=0pt
\tableofcontents % SEM NESAHEJTE!
\parskip=7pt
\newpage % SEM NESAHEJTE!


%--------------------------------------------------------
%|         Zde začíná SAMOTNÁ PRÁCE (text)              |
%--------------------------------------------------------

\chapter*{Úvod} % SEM NESAHEJTE!
\addcontentsline{toc}{chapter}{Úvod} % SEM NESAHEJTE!
%
Zde napište text úvodu (1-3 strany) nebo si text vložte ze samostatného souboru: např. příkazem \texttt{\textbackslash input\{vnitrek\_uvod.tex\}}.\par Text v~úvodu se nerozdělujte na podkapitoly, nýbrž na jednotlivé odstavce. Odstavce od sebe oddělíte prázdným řádkem nebo příkazem \texttt{\textbackslash par}. \par Úvod práce by měl obsahovat vymezení tématu práce (a~důvod výběru tématu), vymezení cíle/cílů práce, smysl cíle, způsob/metodu dosažení cíle (stručný nástin práce) a~předpoklady či omezení práce. Můžete čtenářům také nastínit strukturu práce (tj. co v~které kapitole najdou).
%
%\input{vnitrek_uvod.tex} % vložení souboru, který obsahuje text úvodu (bez nadpisu!)



\chapter{Jak psát závěrečnou práci} % změňte text kapitoly (pokud je \chapter součástí vkládaného souboru, tak tento řádek zakomentujte)
%
\input{vnitrek_kapitola1.tex} % text vkládán ze souboru. Pokud je v souboru uveden i příkaz \chapter{...}, tak ho o 2 řádky výše vymažte!


\chapter*{Závěr}
\addcontentsline{toc}{chapter}{Závěr} % SEM NESAHEJTE!
%
Zde napište text závěru své práce (1-3 strany, nerozdělujte na podkapitoly) nebo jej vložte ze samostatného souboru: např. příkazem \texttt{\textbackslash input\{vnitrek\_zaver.tex\}}. \par Závěr by měl obsahovat shrnutí práce a~zopakovat/zdůraznit, jaké jsou výsledky. Může obsahovat i~náměty na budoucí rozšířené práce. \par V~závěru práce byste neměli hodnotit svou práci -- to udělají členové komise při státní závěrečné zkoušce.
%
%\input{vnitrek_zaver.tex}


%%%%%%%%%%%% SEZNAM POUŽITÝCH ZDROJŮ (LITERATURA) %%%%%%%%%%%%
\clearpage
\addcontentsline{toc}{chapter}{Seznam použitých zdrojů} % SEM NESAHEJTE!
\begin{thebibliography}{99}   
% následující text (2 řádky) zaměňte citacemi svých zdrojů
	\bibitem{odkaz} Autor. \ti{Název knihy}. Město. Nakladatelství. Rok. 
\end{thebibliography}
% formát: ČSN ISO 690. Můžete si to vygenerovat na http://www.citacepro.com (přihlaste se přes odkaz "ČVUT"), umí to vygenerovat TeX
% řazení: abecedně podle autora (resp. provního slova, není-li znám autor)


%%%%%%%%%%%% PŘÍLOHY PRÁCE %%%%%%%%%%%%
\newpage % SEM NESAHEJTE!
\addcontentsline{toc}{chapter}{Přílohy} % SEM NESAHEJTE!
\appendix % SEM NESAHEJTE!


%%%%%%%%%%%% Příloha A (tj. 1. kapitola v rámci příloh) %%%%%%%%%%%%
\chapter{Název přílohy} % zde změňte název přílohy, příp. zakomentujte a vložte soubor, kde je název přílohy + její text (odmažte/zakomentujte text níže + odkomentujte \input{}):
Zde napište text první přílohy nebo jej vložte, např.: \texttt{\textbackslash input\{priloha\_A.tex\}}.
%
%\input{priloha_A.tex} % text vkládán ze souboru, kde je i příkaz \chapter{...}


\end{document} % SEM NESAHEJTE! Konec.
